%%%%%%%%%%%%%%%%%%%%%%%%%%%%%%%%%%%%%%%%%
% Beamer Presentation
% LaTeX Template
% Version 1.0 (10/11/12)
%
% This template has been downloaded from:
% http://www.LaTeXTemplates.com
%
% License:
% CC BY-NC-SA 3.0 (http://creativecommons.org/licenses/by-nc-sa/3.0/)
%
%%%%%%%%%%%%%%%%%%%%%%%%%%%%%%%%%%%%%%%%%

%----------------------------------------------------------------------------------------
%	PACKAGES AND THEMES
%----------------------------------------------------------------------------------------

\documentclass{beamer}

\mode<presentation> {

% The Beamer class comes with a number of default slide themes
% which change the colors and layouts of slides. Below this is a list
% of all the themes, uncomment each in turn to see what they look like.

%\usetheme{default}
%\usetheme{AnnArbor}
%\usetheme{Antibes}
%\usetheme{Bergen}
%\usetheme{Berkeley}
%\usetheme{Berlin}
%\usetheme{Boadilla}
%\usetheme{CambridgeUS}
%\usetheme{Copenhagen}
%\usetheme{Darmstadt}
%\usetheme{Dresden}
%\usetheme{Frankfurt}
%\usetheme{Goettingen}
%\usetheme{Hannover}
%\usetheme{Ilmenau}
%\usetheme{JuanLesPins}
%\usetheme{Luebeck}
\usetheme{Madrid}
%\usetheme{Malmoe}
%\usetheme{Marburg}
%\usetheme{Montpellier}
%\usetheme{PaloAlto}
%\usetheme{Pittsburgh}
%\usetheme{Rochester}
%\usetheme{Singapore}
%\usetheme{Szeged}
%\usetheme{Warsaw}

% As well as themes, the Beamer class has a number of color themes
% for any slide theme. Uncomment each of these in turn to see how it
% changes the colors of your current slide theme.

%\usecolortheme{albatross}
%\usecolortheme{beaver}
%\usecolortheme{beetle}
%\usecolortheme{crane}
%\usecolortheme{dolphin}
%\usecolortheme{dove}
%\usecolortheme{fly}
%\usecolortheme{lily}
%\usecolortheme{orchid}
%\usecolortheme{rose}
%\usecolortheme{seagull}
%\usecolortheme{seahorse}
%\usecolortheme{whale}
%\usecolortheme{wolverine}

%\setbeamertemplate{footline} % To remove the footer line in all slides uncomment this line
%\setbeamertemplate{footline}[page number] % To replace the footer line in all slides with a simple slide count uncomment this line

%\setbeamertemplate{navigation symbols}{} % To remove the navigation symbols from the bottom of all slides uncomment this line
}

\usepackage{graphicx} % Allows including images
\usepackage{booktabs} % Allows the use of \toprule, \midrule and \bottomrule in tables
\definecolor{swcblue}{rgb}{.1641,.1949,.6410}

%----------------------------------------------------------------------------------------
%	TITLE PAGE
%----------------------------------------------------------------------------------------

\title[Short title]{Full Title of the Talk} % The short title appears at the bottom of every slide, the full title is only on the title page

\author{Lynne J. Williams} % Your name
\institute[SWC] % Your institution as it will appear on the bottom of every slide, may be shorthand to save space
{
Software Carpentry \\ % Your institution for the title page
\medskip
\textit{info@software-carpentry.org} % Your email address
}
\date{\today} % Date, can be changed to a custom date
\logo{\includegraphics[height=.75cm]{swc_logo.png}}

\begin{document}

\begin{frame}
\centering
%\titlepage % Print the title page as the first slide
\textcolor{swcblue}{\Huge WHAT TO TAKE HOME}
\end{frame}

%\begin{frame}
%\frametitle{Overview} % Table of contents slide, comment this block out to remove it
%\tableofcontents % Throughout your presentation, if you choose to use \section{} and %\subsection{} commands, these will automatically be printed on this slide as an overview of your presentation
%\end{frame}

%----------------------------------------------------------------------------------------
%	PRESENTATION SLIDES
%----------------------------------------------------------------------------------------

%------------------------------------------------
\section{First Section} % Sections can be created in order to organize your presentation into discrete blocks, all sections and subsections are automatically printed in the table of contents as an overview of the talk
%------------------------------------------------

\subsection{Subsection Example} % A subsection can be created just before a set of slides with a common theme to further break down your presentation into chunks

\begin{frame}
\frametitle{What to take home}
\centering
{\huge Before you  code}\\\pause
{\Large think about what you want to code}\\\pause
{\large (Good for both the code and the programmer)}
\end{frame}

\begin{frame}
\frametitle{What to take home}
\centering
{\huge It never works the first time}\\\pause
{\large And probably won’t the second or third time}
\end{frame}

\begin{frame}
\frametitle{What to take home}
\centering
{\huge Sticking with it is more important than the method/language}
\end{frame}

\begin{frame}
\frametitle{What to take home}
\centering
{\huge Use}\\
{\Huge VERSION CONTROL}
\end{frame}

\begin{frame}
\frametitle{Resources (more at http://software-carpentry.org/biblio.html)}

\includegraphics[width=.15\textwidth]{PracticalProgramming}\quad
\begin{minipage}[b]{.75\textwidth}
\footnotesize
Jennifer Campbell, Paul Gries, Jason Montojo, and Greg Wilson, "{\bfseries Practical Programming: An Introduction to Computer Science Using Python}". Pragmatic Bookshelf, 1934356271, 2009\\
\end{minipage}

\bigskip

\includegraphics[width=.15\textwidth]{Computing4Biologists}\quad
\begin{minipage}[b]{.75\textwidth}
\footnotesize
Steve Haddock and Casey Dunn, "{\bfseries Practical Computing for Biologists}". Sinauer, 0878933913, 2010\\\null\\
\end{minipage}



\end{frame}
%------------------------------------------------



%------------------------------------------------



%----------------------------------------------------------------------------------------

\end{document} 